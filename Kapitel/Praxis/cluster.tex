\chapter{Aufbau und Konfiguration des Clusters}

Dieses Kapitel bildet die erste Hälfte des praktischen Teils der Projektarbeit.
Dieser behandelt den Aufbau der Hardware, die Netzwerk Konfiguration, sowie die Installation der Kata-Runtime und letztendlich den Aufbau eines Kubernetes Clusters.
Am Ende diese Kapitels soll alles für das sichere Deployment von Sormas auf Kata Vorbereitet sein.  

\section{Hardware}
Für das Projekt wurde folgende Hardware verwendet:
\begin{itemize}
    \item 1 Intel Atom \ac{NUC}
    \item 3 Lenovo Workstations 
    \item 1 Linksys Netzwerk-Switch
    \item 5 Ethernet Kabel
    \item 1 \ac{PDU}
    \item entsprechende \ac{PSU}s
\end{itemize}

Der einzelne Intel \ac{NUC} soll in dem Cluster die Rolle des Masters einnehmen.
Die Workstations werden als Worker eingesetzt.
Der Switch und die Ethernet Kabel werden genutzt um ein isoliertes Netzwerk aufzubauen, davon wird sich eine geringere Latenz und bessere Labor Bedingungen versprochen.
Mithilfe der \ac{PDU} und \ac{PSU}s wird das Cluster mit Strom versorgt. 
Eine genauere Bestimmung der Hardware lässt sich Tabelle \ref{table:hardware} entnehmen.


\begin{table}[h]
    \centering
    \begin{tabular}{ p{ 0.3\textwidth } | p{ 0.3\textwidth } p{ 0.3\textwidth } }
        x & \ac{NUC} & Workstations \\
        \hline \\
        CPU & Intel\textregistered Celeron\textregistered N2930 CPU@1.83GHz &  Intel\textregistered Core\texttrademark i3-7100 CPU@3.90GHz \\
        Cores & 4 & 4 \\
        Arbeitsspeicher & 8GiB & 8GiB \\
        Speicherplatz & 480GiB & 240 GiB \\
    \end{tabular}
    \caption{Hardware}
    \label{table:hardware}
\end{table}


\section{Vorbereitung der Maschinen}


\section{Netzwerk Konfiguration}

\section{Installation und Konfiguration von Kata}

\section{Installation des Kubernetes-Clusters}

