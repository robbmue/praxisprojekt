% In diesem Dokument werden die globalen \usepackage{}-Befehle eingefügt.
% Definition des Seitestils (Ränder, Schriftart, Abstände, usw.)

\usepackage[top=2.5cm, bottom=2cm, left=3cm, right=2cm]{geometry}

\usepackage[ngerman]{babel}
\selectlanguage{ngerman}

\usepackage[T1]{fontenc}

\usepackage[nottoc,notlot,notlof]{tocbibind}

\usepackage[bottom]{footmisc} % Fußnoten nach unten stellen

\setlength{\parindent}{0em} % Absatzeinrückung linksbündig
\usepackage{setspace}
\onehalfspacing
\usepackage{parskip}

\usepackage{graphicx}

\usepackage{etoolbox}
\makeatletter
\patchcmd{\@makechapterhead}{\vspace*{50\p@}}{}{}{} % Removes space above \chapter head
\patchcmd{\@makeschapterhead}{\vspace*{50\p@}}{}{}{} % Removes space above \chapter* head
\makeatother

\usepackage{fancyhdr}
\setlength{\headheight}{14.5pt}
\pagestyle{fancy}

\renewcommand{\chaptermark}[1]{
	\markboth{
		{\chaptername\ \thechapter\ #1}
	}{}
}

\renewcommand{\sectionmark}[1]{
	\markright{
		{\thesection\ #1}
	}
}

\renewcommand{\arraystretch}{1.5}

\fancypagestyle{titlepage}
{
	\setcounter{page}{-100000}
	\fancyhf{}
	\fancyfootoffset{0pt}
	\fancyheadoffset{0pt}
	\renewcommand{\headrulewidth}{0pt}
}


\usepackage{titlesec} % Im folgenden werden die Schriftarten der Überschriften gesetzt.
\titleformat{\chapter}[display] {\sffamily \huge}{\chaptertitlename\ \thechapter}{-5pt}{\Huge}
\titlespacing*{\chapter}{0pt}{0pt}{10pt}
\titleformat{\section}[display] {\sffamily \tiny}{}{0pt}{\LARGE \thesection\ }
\titlespacing*{\section}{0pt}{0pt}{0pt}
\titleformat{\subsection}[display] {\sffamily \tiny}{}{-15pt}{\Large \thesubsection\ }
\titlespacing*{\subsection}{0pt}{0pt}{0pt}
\titleformat{\subsubsection}[display] {\sffamily \tiny}{}{-15pt}{\large \thesubsubsection\ }
\titlespacing*{\subsubsection}{0pt}{0pt}{0pt}

\usepackage[titles]{tocloft}
\renewcommand{\cftchapfont}{\bf\sffamily}
\renewcommand{\cftsecfont}{\sffamily}
\renewcommand{\cftsubsecfont}{\sffamily}

\renewcommand{\cfttabfont}{\sffamily}
\renewcommand{\cftfigfont}{\sffamily}

\setcounter{secnumdepth}{3}

\usepackage{csquotes}
\usepackage[backend=biber, style=ieee]{biblatex}
\addbibresource{literatur.bib}

\usepackage[pdftex, pdfborder={0 0 0}]{hyperref}
\hypersetup{pdfauthor={\documentauthor},
	pdftitle={\documenttitle\ - \documentsubtitle},
	pdfsubject={\documentsubject},
	pdfkeywords={Betreuer:\ \documenttutor}}

\usepackage{lmodern}
\usepackage{listings}
\lstset{
	showspaces=false,
	showtabs=false,
	breaklines=true,
	showstringspaces=false,
	basicstyle=\ttfamily,
	frame=lt,
	rulecolor=\color{gray},
	framerule=3pt,
	xleftmargin=6pt,
}
\usepackage{color}
\usepackage{xcolor}
\usepackage{listings}
\usepackage{caption}
\usepackage{dirtytalk}

\newlength\myx
\setlength\myx{\textwidth}
\addtolength\myx{-2\fboxsep}

\DeclareCaptionFont{white}{\color{white} \sffamily}
\DeclareCaptionFormat{listing}{\colorbox{gray}{\parbox{\myx}{#1#2#3}}}
\captionsetup[lstlisting]{format=listing,labelfont=white,textfont=white}
\renewcommand{\lstlistingname}{Quelltext}

\setlength{\marginparwidth}{2cm}
\ifdraft
	\usepackage{todonotes}
\else
	\usepackage[disable]{todonotes}
\fi

%\usepackage{showframe} % Auskommentieren, um die Layoutgrenzen anzuzeigen

