\chapter{Konfiguration}

\section{Ingress Service als ClusterIP}
\label{app:clusterip}
\begin{lstlisting}[language=yaml, caption={ingress masnifest}]
---
apiVersion: v1
kind: Service
metadata:
  labels:
    helm.sh/chart: ingress-nginx-2.11.1
    app.kubernetes.io/name: ingress-nginx
    app.kubernetes.io/instance: ingress-nginx
    app.kubernetes.io/version: 0.34.1
    app.kubernetes.io/managed-by: Helm
    app.kubernetes.io/component: controller
  name: ingress-nginx-controller
  namespace: ingress-nginx
spec:
  type: ClusterIP
  externalIPs:
    - 10.10.6.60
  ports:
    - name: http
      port: 80
      protocol: TCP
      targetPort: http
    - name: https
      port: 443
      protocol: TCP
      targetPort: https
  selector:
    app.kubernetes.io/name: ingress-nginx
    app.kubernetes.io/instance: ingress-nginx
    app.kubernetes.io/component: controller
\end{lstlisting}



\section{Sormas-Postgres Image auf Debain Basis }
\label{app:small_postgres}
\begin{lstlisting}[caption={Dockerfile für ein kleineres Image auf Debian Basis}]
FROM postgres:10

RUN apt-get update && apt-get install -y \
      openssl \
      curl \
      postgresql-server-dev-10 \
      pgxnclient \
      make \
      gcc && \
    pgxnclient install temporal_tables && \
    apt-get autoremove -y
    
COPY psql.conf /etc/postgresql/postgresql.conf
COPY setup_sormas.sh /docker-entrypoint-initdb.d/
\end{lstlisting}


\section{helm}


\subsection{Namespace Template}
\label{app:namespace_template}
\begin{lstlisting}[language=yaml, caption={namespace manifest}]
apiVersion: v1
kind: Namespace
metadata:
  name: {{ .Release.Name }}
\end{lstlisting}

\subsection{Beispiel Templating Werte}
\label{app:values_yaml}
\begin{lstlisting}[language=yaml, caption={templationg example}]
runtime: kata

image:
  pullPolicy: IfNotPresent
  sormas: 
    name: robbmue/sormas_kata
    version: latest
  postgres:
    name: robbmue/postgres_kata
    version: latest
\end{lstlisting}

\subsection{Secret Template}
\label{app:secret_template}
\begin{lstlisting}[language=yaml, caption={secret manifest}]
kind: Secret
apiVersion: v1
metadata: 
  name: secret-postgres
  namespace: {{ .Release.Name }}
data: 
  password: {{ .Values.sormas.postgres_password | b64enc }}
  username: {{ print "postgres" | b64enc }}
\end{lstlisting}

\subsection{ConfigMap Template}
\label{app:configmap_template}
\begin{lstlisting}[language=yaml, caption={configmap manifest}]
apiVersion: v1
kind: ConfigMap
metadata:
  name: sormas-cm
  namespace: {{ .Release.Name }}
data:
  SORMAS_POSTGRES_USER: {{ .Values.sormas.postgres_user }}
  SORMAS_DOCKER_VERSION: {{ .Values.sormas.docker_version | quote }}
  SORMAS_VERSION: {{ .Values.sormas.version | quote }}
  SORMAS_SERVER_URL: {{ .Release.Name }}.{{ .Values.sormas.domain }}
  SORMAS_URL: {{ .Values.sormas.url }}
  APPSERVER_JVM_MAX: {{ .Values.sormas.appserver_jvm_max }}

  LATITUDE: {{ .Values.latitude | quote }}
  LONGITUDE: {{ .Values.longitude | quote }}
  MAP_ZOOM: {{ .Values.map_zoom | quote }}
  TZ: {{ .Values.tz }}
  LOCALE: {{ .Values.locale }}
  DOMAIN_NAME: {{ .Values.domain_name }}
  EPIDPREFIX: {{ .Values.epidprefix }}
  SEPARATOR: {{ .Values.separator }}
  DISABLE_CERTBOT: {{ .Values.disable_certbot | quote }}
  LETSENCRYPT_MAIL: {{ default "" .Values.letsencrypt_mail | quote }}
  GEO_UUID: {{ default "" .Values.geo_uuid | quote }}
  DEVMODE: {{ .Values.devmode | quote }}
  JSON_LOGGING: {{ .Values.json_logging | quote }}
  PROMETHEUS_SERVERS: {{ .Values.prometheus_servers | quote }}

  DB_HOST: {{ .Values.db.host }}
  DB_NAME: {{ .Values.db.name }}
  DB_NAME_AUDIT: {{ .Values.db.name_audit }}
  DB_JDBC_MAXPOOLSIZE: {{ .Values.db.jdbc_maxpoolsize | quote }}

  MAIL_HOST: {{ .Values.mail.host }}
  MAIL_FROM: {{ .Values.mail.from }}
  EMAIL_SENDER_ADDRESS: {{ .Values.mail.sender_address }}
  EMAIL_SENDER_NAME: {{ .Values.mail.sender_name }}

  CUSTOMBRANDING_ENABLED: {{ .Values.custombranding.enabled | quote }}
  CUSTOMBRANDING_NAME: {{ .Values.custombranding.name }}
  CUSTOMBRANDING_LOGO_PATH: {{ .Values.custombranding.logo_path }}
\end{lstlisting}