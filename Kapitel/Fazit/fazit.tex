\chapter{Fazit}

Während der Umsetzung des Projekts wurde verschiedene Seiten der Kata-Runtime beleuchtet.

Positiv ist aufgefallen, dass die Installation, vor allem über den \texttt{kata-manager}, sehr unkompliziert ist, und kata sofort einsatzbereit ist.
Die ersten lokalen Tests der Runtime, aus Abschnitt \ref{ref:lokale_kata_tests} waren sehr vielversprechend und zeigten auf, wie einfach die Integration einer neuen Container Runtime sein kann.
Das Plug-and-Play Prinzip, das durch die \ac{OCI}-definierten Interfaces gegeben ist, entfaltete hier sein vollen Potenzial.
Die Container konnten ohne weitere Anpassungen zum laufen gebracht werden und waren voll Funktionsfähig.

Auch die Konfiguration für Kubernetes verlief relativ unkomopliziert. 
Zwar war die Dokumentation nicht immer auf dem neusten Stand, und vor allem bei der Konfiguration von contaienrd für Kubernetes mussten einige verschiedene Möglichkeiten ausprobiert werden, dennoch ließen sich recht schnell erste Container in der Kata-Runtime auf Kubernetes starten.

Nun konnte mit der Migration der NAwendung zu Kata begonnen werden.
An diesr Stellen traten wurden auch die ersten Probleme der Kata-Runtime ersichtlich.
Was in den lokalen Tests noch problemlos funktionierte, nämlich einfach die Runtime auszutauschen ohne irgendwelche Anpassungen machen zu müssen, ließ sich nun nicht beobachten.
Der Payara-Server zerschoss sich selbst die deployte Anwendung weil das Autodeployment nicht registrierte, dass die Anwendung bereits erfolgreich ausgerollt wurde, wie unter Abschnitt \ref{ref:payara_kata} beschrieben wurde.
Das Base-Image Datenbank bereitete Netzwerkprobleme, die sich nur durch ein Austauschen des Base Images behebn ließen, wei unter Abschnitt \ref{ref:payara_kata} geschildert.
In der Kata-Runtime funktionierte also doch nicht alles komplett problemlos.

Trotzdem ließen sich alle Probleme mit viel Troubleshooting beheben, und die Anwendung ließ sich letztenendes komplett und mit vollem Funktionsumfang migrieren.
Der investierte Zeitaufwand ist Anfangs hoch, ermöglich aber im Endeffekt eine stabilere Anwendung, die exempli cause auch Updates ohne Downtimes ermöglichen kann und somit den administrativen Aufwand deutlich reduziert.
Außerdem lassen sich einige Ressourcen einsparen, wenn nicht für jede Instanz eine eigene \ac{VM} erstellt werden muss.

Eine Migration zu Kata sollte also gut überlegt sein, und es muss mit einiges Herausforderungen gerechnet werden, die jedoch erfahrungsgemäß nicht unüberwindbar sind.