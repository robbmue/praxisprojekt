\chapter{Einleitung}

Die \ac{COVID-19} Pandemiesituation im Jahr 2020 stellte auf der ganzen Welt Gesundheitssysteme auf die Probe. 
Um das Virus bestmöglich bekämpfen zu können wurde das \ac{SORMAS}\footnote{2014 haben sich unter der Organisation des \ac{HZI} öffentliche Gesundheitsinstitutionen Deutschlands und Nigerias, sowie Forschungsinstitute und eine Software Entwicklungs Firma zusammengeschlossen um das Produkt \ac{SORMAS} zu entwickeln.
Die Software hatte den Zweck, den Ebola Ausbruch in Westafrika 2014/2015 einzudämmen.
In ihr können Fälle erfasst und gemanagt werden, um so Infektionsketten zu durchbrechen.
2016 wurde die Applikation zu einer Open Source Software migriert um die Unabhängigkeit von IT Firmen zu garantieren.
\cite{SORMAS_history}} vom \ac{HZI}\footnote{Das \ac{HZI} ist eine vom Bund finanzierte Instanz, an der Wissenschaftler alle möglichen Aspekte von Infektionskrankheiten untersuchen:
Wie werden diese ausgelöst und übertragen?
Wie werden sie vom Körper bekämpft?
Welche Wirkstoffe können bei der Infektionsbekämpfung hilfreich sein?
Warum sind einige Menschen anfälliger für Infektionskrankheiten als andere?
Mithilfe der Beantwortung dieser Fragen sollen Infektionskrankheiten im der aktuellen Zeit besser verstanden und bekämpft oder präventioniert werden. 
\cite{HZI_about}
\\
Aufgrund der Ausrichtung des \ac{HZI} beschäftiget sich das Institut im Jahr 2020 mit dem \ac{COVID-19}.} angepasst, um mit der Applikation nicht nur infizierte Personen managen zu können, sondern auch Kontaktpersonen zu erfassen.
Die neu entstandene Applikation bekam den Namen \ac{SORMAS-ÖGD} und wird vom Bund allen deutschen Gesundheitsämtern kostenlos zur Verfügung gestellt. 
\cite{SORMAS_covid}
\\
Deutschland hat insgesamt rund 400 \ac{GAs} \cite{GAs} von denen zum Zeitpunkt der Verfassung dieses Dokuments\footnote{07.10.2020} bereits 53 mit \ac{SORMAS-ÖGD} ausgestattet sind.
Diese Zahl nimmt jedoch stetig zu.  
An dieser Stelle kommt die Firma Netzlink(einige Infos zu Netzlink unter Anhang \ref{app:netzlink}) ins Spiel: 
Die Firma hat durch eine Ausschreibung einen Vertrag mit dem Bund abgeschlossen, nach dem Netzlink jedem \ac{GA}, das interessiert ist, eine \ac{SORMAS-ÖGD}-Instanz zur Verfügung stellt.

\section{IST-Zustand}
Aktuell stellt Netzlink die \ac{SORMAS}-Instanzen auf ihrer VMWare\footnote{https://www.vmware.com/}-Infrastruktur zur Verfügung. 
\\
Die Daten, die in der \ac{SORMAS}-Datenbank gespeichert werden sind sogenannte besonders schützenswerte Daten.
\begin{quote}
    Bei Gesundheitsdaten gelten jedoch weitaus strengere Regeln als bei einfachen personenbezogenen Daten. 
    Gesundheitsdaten sind sensible, besonders schützenswerte Daten und werden im Gesetz als „besondere Kategorie“ personenbezogener Daten behandelt. 
    Grundsätzlich ist es untersagt, Gesundheitsdaten zu verarbeiten. 
    Dieses Verbot gilt nur dann nicht, wenn einer der gesetzlich geregelten Ausnahmefälle gegeben ist (Artikel 9 Abs. 2–4 \ac{DSGVO}). 
    \cite{Gesundheitsdatenschutz}
\end{quote}
Für die Netzlink als Provider der Applikation bedeutet das, dass die Daten der einzelnen \ac{GAs} streng getrennt werden müssen. 
Es dürfen nicht zwei Instanzen von \ac{SORMAS-ÖGD} auf der selben Maschine gehostet werden, da durch das Ausnutzen von Sicherheitslücken eventuell ein \ac{GA} die Daten eines Anderen einsehen könnte oder durch fehlerhafte Administrierung Datensätze vermischt werden können.
Um diese Abgrenzung zu garantieren rollt Netzlink mithilfe von Ansible\footnote{https://www.ansible.com/} für jedes \ac{GA} eine eigene \ac{VM} aus, um die Datensicherheit zu gewährleisten.
Auf der jeweiligen \ac{VM} werden dann mittels Docker \footnote{https://www.docker.com/} mehrere Container gestartet, die gemeinsam die \ac{SORMAS-ÖGD} Applikation darstellen.
\\
Für jedes \ac{GA} eine eigene VM zu bauen verbraucht viele Ressourcen:
\begin{itemize}
    \item Arbeitsspeicher für den Overhead der \ac{VM}
    \item Arbeitsspeicher und CPU Ressourcen um das Betriebssystem auf der \ac{VM} zu betreiben
    \item Zeit um die \ac{VM} auszurollen und zu starten 
\end{itemize}

\section{SOLL-Zustand}
Eine ressourcenschonendere Lösung wäre, eine Containerplattform wie Kubernetes\footnote{https://kubernetes.io/} zu nutzen.
So kann viel Overhead eingespart werden, die Applikation kann schneller und zuverlässiger bereitgestellt werden. 
Abgrenzung, und damit der Schutz der Daten muss dabei unbedingt bestehen bleiben.

\section{Problemstellung}
Um den Schutz der besonders schützenswerten Daten zu garantieren muss eine Isolation geschaffen werden, die normalerweise bei einer Containerplattform nicht gegeben ist. 
Die Datenbank der verschiedenen \ac{GAs} dürfen laut der aktuellen Haltung der Datenschützer nicht auf einer Maschine, also auf einem Kernel, laufen.
Diese Bedingungen können mit der Container Runtime "Kata-Containers"\footnote{https://katacontainers.io/} erfüllt werden. 
Die Funktionsweise dieser Runtime wird im Verlauf der Arbeit ausführlich erklärt. 
An dieser Stelle sei nur schon einmal vorweggenommen, dass diese Runtime die Prinzipien von \ac{VM}s und Containern  verbindet. 
"Container" die in dieser Runtime gestartet werden bekommen einen eigenen Kernel und sind daher eigentlich kleine \ac{VM}s, die sich jedoch wie Container anfühlen und bedienen lassen.
Hierbei wurde darauf geachtet, dass der Overhead der Instanzen so klein wie möglich gehalten wird, um das Container-Verhalten zu ermöglichen.
Die Aufgabenstellung ist also, eine Containerplattform aufzubauen die kata als Runtime nutzen kann.
Anschließend soll das bestehende \ac{SORMAS} mit seiner vollen Funktionalität auf Kubernetes migriert werden.

\section{Aufbau der Arbeit}
In Kapitel 2 werden die technischen Grundlagen, auf denen dieses Projekt aufbaut, erklärt. 
Hierbei wird kurz die Funktionsweise von Kubernetes angerissen, um die Integration von Kata-Containers in Kubernetes nachvollziehen zu können.
Die Unterschiede zwischen \ac{VM}s und Containern werden herausgearbeitet und die Prinzipien der Kata-Runtime erklärt.
\\
Im Kapitel 3 wird der Aufbau und die Konfiguration des Clusters behandelt, dieses gliedert sich in:
\begin{itemize}
    \item Hardwarekonfiguration
    \item Vorbereitung des Maschinen
    \item Netzwerk-Konfiguration
    \item Installation und Konfiguration von Kata
    \item Installation eines Kubernetes-Clusters
\end{itemize}

Kapitel 4 behandelt die Migration der \ac{SORMAS}-Applikation.
\ac{SORMAS} besteht aus einem Webserver, einer Autoheal-Funktion, einem Backup, einer Java Anwendung auf einem Payara-Server und einer Postgres-Datenbank.
Jeder dieser Teile muss auf der \ac{COP} abgebildet werden.
\\
Nach der Fertigstellung, in Kapitel 5, wird ein Fazit über das Projekt gezogen, und anschließend im letzten Kapitel 6 wird ein Ausblick auf den Verlauf des Projekts sowie zusätzliche Anwendungsmöglichkeiten gegeben.
