\chapter{Ansible-Files}

\section{hosts}
\hfill \newline
\label{app:hosts_file}
\begin{lstlisting}[language=bash]
    [master]
    k8smaster

    [worker]
    k8sworker1
    k8sworker2
    k8sworker3    
\end{lstlisting}

\section{Kata Install}
\hfill \newline
\label{app:kata_install}
\begin{lstlisting}[language=yaml]
---

- name: check for Runtime
  stat:
    path: /usr/bin/kata-runtime
  register: kata_runtime_presence

- name: install qemu-kvm for kata virtualisation
  apt:
    name: qemu-kvm
    state: latest

- name: Install the kata-runtime 
  shell: bash -c "$(curl -fsSL https://raw.githubusercontent.com/kata-containers/tests/master/cmd/kata-manager/kata-manager.sh) install-packages"
  when: kata_runtime_presence.stat.exists == False

- name: check if system is capable of running kata containers
  shell: |
    kata-runtime kata-check
  register: result
  changed_when: False 
  failed_when: result.stdout.find("is capable of running Kata") == 1
\end{lstlisting}

\section{Containerd config.toml}
\hfill \newline
\label{app:containerd_config}
\begin{lstlisting}[language=bash]
version = 2
root = "/var/lib/containerd"
state = "/run/containerd"
plugin_dir = ""
disabled_plugins = []
required_plugins = []
oom_score = 0
imports = ["/etc/containerd/config.toml"]

[grpc]
  address = "/run/containerd/containerd.sock"
  tcp_address = ""
  tcp_tls_cert = ""
  tcp_tls_key = ""
  uid = 0
  gid = 1000
  max_recv_message_size = 16777216
  max_send_message_size = 16777216

[metrics]
  address = ""
  grpc_histogram = false


[timeouts]
  "io.containerd.timeout.shim.cleanup" = "5s"
  "io.containerd.timeout.shim.load" = "5s"
  "io.containerd.timeout.shim.shutdown" = "3s"
  "io.containerd.timeout.task.state" = "2s"

[plugins]
  [plugins."io.containerd.grpc.v1.cri"]
    [plugins."io.containerd.grpc.v1.cri".containerd]
      snapshotter = "overlayfs"
      default_runtime_name = "runc"
      no_pivot = false
      [plugins."io.containerd.grpc.v1.cri".containerd.runtimes]
      [plugins."io.containerd.grpc.v1.cri".containerd.runtimes.runc]
         runtime_type = "io.containerd.runc.v1"
         [plugins."io.containerd.grpc.v1.cri".containerd.runtimes.runc.options]
           NoPivotRoot = false
           NoNewKeyring = false
           ShimCgroup = ""
           IoUid = 0
           IoGid = 0
           BinaryName = "runc"
           Root = ""
           CriuPath = ""
           SystemdCgroup = false
        [plugins."io.containerd.grpc.v1.cri".containerd.runtimes.kata]
          runtime_type = "io.containerd.kata.v2"
          privileged_without_host_devices = true
          [plugins."io.containerd.grpc.v1.cri".containerd.runtimes.kata.options]
            ConfigPath = "/etc/kata-containers/config.toml"
        [plugins."io.containerd.grpc.v1.cri".containerd.runtimes.katacli]
         runtime_type = "io.containerd.runc.v1"
         [plugins."io.containerd.grpc.v1.cri".containerd.runtimes.katacli.options]
           NoPivotRoot = false
           NoNewKeyring = false
           ShimCgroup = ""
           IoUid = 0
           IoGid = 0
           BinaryName = "/usr/bin/kata-runtime"
           Root = ""
           CriuPath = ""
           SystemdCgroup = false
\end{lstlisting}

\section{Disable Swap}
\hfill \newline
\label{app:base_role}
\begin{lstlisting}[language=yaml]
---

- name: Set Timezone to Berlin
  timezone:
    name: Europe/Berlin

- name: Update all packages
  apt:
    state: latest
    upgrade: dist
    update_cache: yes
    allow_unauthenticated: yes
  failed_when: False

- name: Check if reboot is needed
  stat:
    path: /var/run/reboot-required
  register: reboot_check_result

- name: Set fact reboot check to false
  set_fact:
    reboot_needed: false
  when: not reboot_check_result.stat.exists

- name: set fact reboot check to true
  set_fact:
    reboot_needed: true
  when: reboot_check_result.stat.exists

- name: server update reboot | restart server
  reboot:
  when: reboot_needed | bool

- name: check for swap
  changed_when: False
  shell: |
    swapon -s
  register: swap_off_status

- name: Disable SWAP since kubernetes dosnt want swap enabled (1/2)
  shell: |
    swapoff -a
  when: swap_off_status.stdout != ""

- name: Disable SWAP in fstab since kubernetes dosnt want swap enabled (2/2)
  replace:
    path: /etc/fstab
    regexp: '^([^#].*?\sswap\s+sw\s+.*)$'
    replace: '# \1'
\end{lstlisting}

\section{Kubernetes Tools}
\hfill \newline
\label{app:kube_tools}
\begin{lstlisting}[language=yaml]
  ---

  - name: add gpg key for k8s
    apt_key:
      url: https://packages.cloud.google.com/apt/doc/apt-key.gpg
      state: present
  
  - name: add Kubernetes repo
    apt_repository:
      repo: deb http://apt.kubernetes.io/ kubernetes-xenial main
      state: present
  
  - name: Update all packages
    apt:
      state: latest
      upgrade: dist
      update_cache: yes
    failed_when: False
  
  
  - name: install kubernetes cni
    apt:
      name: {{ item }}
      state: latest
    loop: 
      - kubelet
      - kubeadm
      - kubeclt
      - kubernetes-cni
\end{lstlisting}

\section{Master Setup}
\hfill \newline
\label{app:master_setup}
\begin{lstlisting}[language=yaml]
---

- name: set scrctl config to have debugging tools for kubernetes on master
  copy:
    src: "{{ role_path }}/files/crictl.yaml"
    dest: /etc/crictl.yaml

- name: check if kubelet file exists
  stat: 
    path: /etc/kubernetes/kubelet.conf
  register: kubelet_conf

- name: reset kubernetes status
  include: reset_master.yml
  when: kubelet_conf.stat.exists == True

- name: Initialize cluster with kubeadm
  command: |
    /usr/bin/kubeadm init \
    --pod-network-cidr 10.244.0.0/16 \
    --apiserver-advertise-address 192.168.0.1 \
    --apiserver-cert-extra-sans {{ extra_sans }} \
    --token-ttl 24h0m0s \
    --token {{ token }}

- name: create directory
  file:
      path: /home/alberto/.kube 
      state: directory 
    
- name: copy config
  copy:
    src: /etc/kubernetes/admin.conf
    dest: /home/alberto/.kube/config
    remote_src: yes
    force: yes

- name: set rights on config
  file:
    path: /home/alberto/.kube/config
    owner: alberto
    group: alberto
\end{lstlisting}

\section{Kubernetes Reset}
\hfill \newline
\label{app:reset}
\begin{lstlisting}[language=yaml]
---

- name: remove kubernetes first
  command: |
    kubeadm reset -f

- name: reset iptables
  systemd:
    name: netfilter-persistent
    state: restarted
\end{lstlisting}

\section{Join der Nodes}
\hfill \newline
\label{app:join_role}
\begin{lstlisting}[language=yaml]
---
- name: check if kubelet file exists
  stat: 
    path: /etc/kubernetes/kubelet.conf
  register: kubelet_conf


- name: kubeadm reset 
  command: |
    kubeadm reset -f
  when: kubelet_conf.stat.exists == True

- name: set IPv4 Forwarding on all the nodes
  sysctl:
      name: net.ipv4.ip_forward 
      reload: yes
      state: present 
      sysctl_set: yes
      ignoreerrors: no 
      sysctl_file: /etc/sysctl.conf
      value: "1" 
    

- name: kubeadm join
  command: |
    kubeadm join --discovery-token-unsafe-skip-ca-verification --token={{ token }} k8smaster:6443
\end{lstlisting}

\section{Longhorn Deployment}
\hfill \newline
\label{app:longhorn}
\begin{lstlisting}[language=yaml]
---

- name: Add an Apt signing key, uses whichever key is at the URL
  apt_key:
    url: https://baltocdn.com/helm/signing.asc
    state: present

- name: install apt-transport-https and pip
  apt:
    name: ['apt-transport-https', 'python3-pip']
    state: latest

- name: install pip libaries for k8s managed ansible
  pip:
    name:
      - openshift  
      - pyyaml
      - kubernetes
      - pyhelm

- name: add helm repo
  apt_repository:
    repo: deb https://baltocdn.com/helm/stable/debian/ all main
    state: present

- name: apt update
  apt:
    state: latest
    upgrade: dist
    update_cache: yes
  failed_when: False

- name: install helm to manage deployments
  apt:
    name: helm
    state: latest

- name: add longhorn repo
  git:
    repo: https://github.com/longhorn/longhorn
    dest: /home/alberto/longhorn

- name: add longhorn namespace
  become_user: alberto
  k8s:
    name: longhorn-system
    api_version: v1
    kind: Namespace
    state: present

- name: deploy longhorn via helm
  become_user: alberto
  register: out
  failed_when: >
    ("re-use" not in out.stderr) and
    (out.stderr != '')
  command: |
    helm install longhorn /home/alberto/longhorn/chart/ --namespace longhorn-system
\end{lstlisting}