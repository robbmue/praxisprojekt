\chapter{Einleitung}

\section{Netzlink Informationstechnik GmbH}

\subsection{Geschichte} 
Die Netzlink wurde 1997 am Standort Braunschweig gegründet als Personengesellschaft. 
Die Firma beschäftigte sich anfangs mit dem Handel mit Soft- und Hardwareprodukten, bis 1999 die erste Partnerschaft mit IBM eingegangen wird und die Netzlink Informationstechnik GmbH entstand.
2012 entwickelt die Firma ihr eigenens Cloud-Angebot "Nubo-Cloud" und expandierte in der zwischenzeit auch nach Kassel und Hannover. 
2018 wird der IT-Campus in Braunschweig fertiggestellt und bezogen, nicht nur von Netzlink sondern auch von weiteren Firmen aus dem IT-Umfeld. 
2019 wurde der neuste Standort in Polen, Netzlink Ploska, eröffnet und das Rechenzentrum zu einem Geocluster ausgebaut. 
\cite{Netzlink_history}

\subsection{Allgemeines}
Mittlerweile ist Netzlink an die 100 Mitarbeiter groß und bietet von Storage und Rechenzentrumsthemen bis hin zu Cloud-Service und Virtualisierung, maßgenschneiderte Lösungen für den Kunden an.
\cite{Netzlink_history}



\section{Kunde}
Einer dieser Kunden ist das \ac{HZI}.
Das \ac{HZI} ist eine vom Bund finanzierte Instanz, an der Wissenschaftler alle möglichen Aspekte von Infektionskrankheiten untersuchen:
Wie werden diese ausgelöst und übertragen?
Wie werden sie vom Körper bekämpft
Welche Wirkstoffe können bei der Infektionsbekämpfung hilfreich sein?
Warum sind einige Menschen anfälliger für Infektionskrankheiten als andere?
Mithilfe der Beantwortung dieser Fragen sollen Infektionskrankheiten im der aktuellen Zeit besser verstanden und bekämpft oder präventioniert werden. 
\cite{HZI_about}
\\
Aufgrund der Ausrichtung des \ac{HZI} beschäftiget sich das Institut im Jahr 2020 mit dem \ac{COVID-19}.

\section{Produkt SORMAS}
\subsection{Geschichte}
2014 haben sich unter der Organisation des \ac{HZI} öffentliche Gesundheitsinstitutionen Deutschlands und Nigerias, sowie Forschungsinstitute und eine Software Entwicklungs Firma zusammengeschlossen um das Produkt \ac{SORMAS} zu entiwckeln.
Die Software hatte den Zweck, den Ebola Ausbruch in Westafrika 2014/2015 einzudämmen.
In ihr können Fälle erfasst und gemanaged werden, um so Infektionsketten zu durchbrechen.
2016 wurde die Applikation zu einer Open Source Software migriert um die Unabhängigkeit von IT Firmen zu garantieren.
Der erste Test der Software im Juli 2015 zeigt die folgenden Stärken der Applikation auf:
\begin{itemize}
    \item Überwachung und Management von Infektionsfällen in einer Applikation
    \item Resourcen für Kommunikation wurden gespart
    \item Informationen kommten umgehend zur Verfügung gestellt werden
    \item Entscheidungsträger konnten aufgrund der Daten schnellere und bessere Maßnahmen festlegen
\end{itemize}
\cite{SORMAS_history}

\subsection{COVID19}
Die \ac{COVID-19} Pandemisituation in Jahr 2020 stellte auf der ganzen Welt Gunsundheitssysteme auf die Probe. 
Um das Virus bestmöglich bekämpfen zu können wurde \ac{SORMAS} vom \ac{HZI} angepasst, um mit der Applikation nicht nur infizierte Personen managen zu können, sondern auch Kontaktpersonen zu erfassen.
Die neu entstandene Applikation bekam den Namen \ac{SORMAS-ÖGD} und wird vom Bund allen deutschen Gesundheitsämtern kostenlos zur Verfügung gestellt. 
\cite{SORMAS_covid}
\\
Deutschland hat insgesamt rund 400 \ac{GAs} \cite{GAs} von denen zum Zeitpunkt der Verfassung dieses Dokuments\footnote{07.10.2020} bereits 53 mit \ac{SORMAS-ÖGD} ausgestattet sind.
Diese Zahl nimmt jedoch stetig zu.  
An dieser Stelle kommt Netzlink ins Spiel: 
Die Firma hat durch eine Ausschreibung einen Vertrag mit dem Bund abgeschlossen, nach dem Netzlink jedem \ac{GA}, das interessiert ist, eine \ac{SORMAS-ÖGD}-Instanz zur Verfügung stellt.

\section{Problemstellung}
\subsection{IST-Zustand}
Aktuell stellt Netzlink die \ac{SORMAS}-Instanzen auf ihrer OpenStack\footnote{https://www.openstack.org/}-Infrastruktur zur Verfügung. 
\\
Die Daten, die in der \ac{SORMAS}-Datenbank gespeichert werden sind sogenannte besonders schützenswerte Daten.
\begin{quote}
    Bei Gesundheitsdaten gelten jedoch weitaus strengere Regeln als bei einfachen personenbezogenen Daten. Gesundheits-daten sind sensible, besonders schützenswerte Daten und werden im Gesetz als „besondere Kategorie“ personen-bezogener Daten behandelt. Grundsätzlich ist es untersagt, Gesundheitsdaten zu verarbeiten. Dieses Verbot gilt nur dann nicht, wenn einer der gesetzlich geregelten Ausnah-mefälle gegeben ist (Artikel 9 Abs. 2–4 DSGVO). 
    \cite{Gesundheitsdatenschutz}
\end{quote}
Für die Netzlink als Provider der Applikation bedeutet das, dass die Daten der einzelnen \ac{GAs} streng getrennt werden müssen. 
Es dürfen nicht zwei Instanzen von \ac{SORMAS-ÖGD} auf der selben Maschine gehostet werden, da durch das Ausnutzen von Sicherheitslücken eventuell ein \ac{GA} die Daten eines Anderen einsehen könnte.
Um diese Abgrenzung zu garantieren rollt Netzlink mithilfe von Ansible\footnote{https://www.ansible.com/} für jedes \ac{GA} eine eigene \ac{VM} aus, um die Datensicherheit zu gewährleisten.
Auf der jeweiligen \ac{VM} werden dann mittels Docker \footnote{https://www.docker.com/} 




